
% Seperation between lines
\pgfmathsetlengthmacro{\wavesep}{2.0em}

% Height of a waveform
\pgfmathsetlengthmacro{\waveheight}{1.2em}

% Width of a brick (half a cycle)
\pgfmathsetlengthmacro{\wavewidth}{1em}

% Width of the slant on slanted signal changes
\pgfmathsetlengthmacro{\transitionwidth}{0.3em}

% Width of the curve on slow signal changes (e.g. to z)
\pgfmathsetlengthmacro{\curvedtransitionwidth}{1.0em}

% Special signal styles
\tikzset{wave x/.style={pattern=north east lines}}
\tikzset{wave bus/.style={fill=white}}
\tikzset{wave busyellow/.style={fill=yellow!25!white}}
\tikzset{wave busorange/.style={fill=orange!25!white}}
\tikzset{wave busblue/.style={fill=blue!25!white}}
\tikzset{wave pulled/.style={dotted}}

% Initialise pointer
% \coordinate (last waveform);

% Label for a signal. Arguments:
%  #1: The human-readable label string
\newcommand{\signallabel}[1]{
   \node [ anchor=east
         , left=0 of last waveform
         , minimum height=\waveheight
         ]
         {#1};
}

% Advance the "last brick" coordinate to the next position
%  #1: Width of a brick
\newcommand{\advancebrick}[1]{
   \coordinate (last brick) at ([shift={(#1*\wavewidth,0)}]last brick);
}

% Add the label for a bus
%  #1: Label coordinate
%  #2: Label text
\newcommand{\busdata}[2]{
   \node [ inner sep=0
         , minimum height=\waveheight
         , anchor=mid
         , font=\footnotesize
         ]
         at ([shift={(0.5*\transitionwidth,0)}]#1)
         {#2};
}

% Define a clip which will truncate a waveform at the left-hand side
%  #1: Width of a brick
%  #2: Truncation offset (in bricks)
%  #3: Number of bricks in waveform
\newcommand{\truncatewaveform}[3]{
   \coordinate (last brick) at ([shift={(-#2*\wavewidth,0)}]last brick);
   \clip ([shift={(#2*\wavewidth, 0.6*\waveheight)}]last brick)
         rectangle ++(#3*#1*\wavewidth-#2*\wavewidth,-1.2*\waveheight);
}

% Spacer Brick Overlay
%  #1: brick width
\newcommand{\brickspaceroverly}[1]{
   \pgfmathsetlengthmacro{\spacerheight}{1.2*\waveheight}
   \pgfmathsetlengthmacro{\spacerwidth}{\transitionwidth}
   \pgfmathsetlengthmacro{\spacergap}{0.7*\transitionwidth}

   % Mask off the waveform
   \fill [fill=white]
         ([xshift=-#1*\wavewidth-0.5*\spacergap-0.5*\spacerwidth, yshift=-0.5*\spacerheight]last brick)
      .. controls +(0.8*\spacergap, 0) and +(-0.8*\spacergap, 0)
      .. ++(\spacerwidth, \spacerheight)
      -- ++(\spacergap,0)
      .. controls +(-0.8*\spacergap, 0) and +(0.8*\spacergap, 0)
      .. ++(-\spacerwidth, -\spacerheight)
       ;

   % Lines
   \draw ([xshift=-#1*\wavewidth-0.5*\spacergap-0.5*\spacerwidth, yshift=-0.5*\spacerheight]last brick)
      .. controls +(0.8*\spacergap, 0) and +(-0.8*\spacergap, 0)
      .. ++(\spacerwidth, \spacerheight)
         ++(\spacergap,0)
      .. controls +(-0.8*\spacergap, 0) and +(0.8*\spacergap, 0)
      .. ++(-\spacerwidth, -\spacerheight)
       ;

}

% Generic mid-bus brick
%  #1: brick width
%  #2: brick style
\newcommand{\brickbus}[2]{
   \fill [#2]
         ([yshift= 0.5*\waveheight]last brick)
      -- ++(#1*\wavewidth,0)
      -- ++(0,-\waveheight)
      -- ++(-#1*\wavewidth,0)
      -- cycle
       ;
   \draw ([yshift= 0.5*\waveheight]last brick) -- ++(#1*\wavewidth,0)
         ([yshift=-0.5*\waveheight]last brick) -- ++(#1*\wavewidth,0)
       ;

   \advancebrick{#1}
}

% Generic mid-bit brick
%  #1: brick width
%  #2: Line style
%  #3: Start line offset (from bottom)
%  #4: End line offset (from bottom)
%  #5: Add arrow on transition
\newcommand{\brickbit}[5]{
   \pgfmathsetlengthmacro{\vstart}{(#3-0.5)*\waveheight}
   \pgfmathsetlengthmacro{\vend}{(#4-0.5)*\waveheight}

   % The bit value
   \draw [#2]
         ([yshift=\vend]last brick)
      |- ([yshift=\vstart, xshift=#1*\wavewidth]last brick);

   % Arrow (if required)
   \ifthenelse{\equal{#5}{1}}{
      \path [decoration={ markings
                        , mark=at position 0.5 with {\arrow{>}}
                        }
            , postaction={decorate}
            ]
            ([yshift=\vend]last brick)
         -- ([yshift=\vstart]last brick);
   }{}

   \advancebrick{#1}
}

% Generic bit-glitch brick
%  #1: brick width
%  #2: Style
%  #3: Edge start (offset from bottom)
\newcommand{\brickbitglitch}[3]{
   \pgfmathsetlengthmacro{\voffset}{(#3-0.5)*\waveheight}

   \draw [#2]
         ([yshift=\voffset]last brick)
      -- ([xshift=0.5*\transitionwidth]last brick)
      -- ([yshift=\voffset, xshift=\transitionwidth]last brick)
      -- ++(#1*\wavewidth - \transitionwidth,0);

   \advancebrick{#1}
}

% Generic sharp bit-to-bit transition
%  #1: brick width
%  #2: Brick style
%  #3: Old bit (offset from bottom)
%  #4: New bit (offset from bottom)
%  #5: Include arrow
\newcommand{\bricksharpbittobit}[5]{
   \pgfmathsetlengthmacro{\vstart}{(#3-0.5)*\waveheight}
   \pgfmathsetlengthmacro{\vend}{(#4-0.5)*\waveheight}

   % The transition and new bit
   \draw [#2]
         ([yshift=\vstart]last brick)
      -- ([yshift=\vend]last brick)
      -- ++(#1*\wavewidth,0);

   % Arrow (if required)
   \ifthenelse{\equal{#5}{1}}{
      \ifthenelse{\not\equal{#3}{#4}}{
         \path [decoration={ markings
                           , mark=at position 0.5 with {\arrow{>}}
                           }
               , postaction={decorate}
               ]
               ([yshift=\vstart]last brick)
            -- ([yshift=\vend]last brick);
      }{}
   }{}

   \advancebrick{#1}
}

% Generic sharp bus-to-bit transition
%  #1: brick width
%  #2: Brick style
%  #3: New bit (offset from bottom)
%  #4: Include arrow
\newcommand{\bricksharpbustobit}[4]{
   \pgfmathsetlengthmacro{\vstart}{((1-#3)-0.5)*\waveheight}
   \pgfmathsetlengthmacro{\vend}{(#3-0.5)*\waveheight}

   % The transition and new bit
   \draw [#2]
         ([yshift=\vstart]last brick)
      -- ([yshift=\vend]last brick)
      -- ++(#1*\wavewidth,0);

   % Arrow (if required)
   \ifthenelse{\equal{#4}{1}}{
      \path [decoration={ markings
                        , mark=at position 0.5 with {\arrow{>}}
                        }
            , postaction={decorate}
            ]
            ([yshift=\vstart]last brick)
         -- ([yshift=\vend]last brick);
   }{}

   \advancebrick{#1}
}

% Generic soft bit-to-bit transition
%  #1: brick width
%  #2: Last brick style
%  #3: This brick style
%  #4: Last bit (offset from bottom)
%  #5: New bit (offset from bottom)
\newcommand{\bricksmoothbittobit}[5]{
   \pgfmathsetlengthmacro{\vstart}{(#4-0.5)*\waveheight}
   \pgfmathsetlengthmacro{\vend}{(#5-0.5)*\waveheight}

   % Scale the transition depending on the magnitude of the level change
   \pgfmathsetlengthmacro{\thistranswidth}{abs(#4-#5)*\transitionwidth}
   \pgfmathsetlengthmacro{\thisleadwidth}{\transitionwidth - \thistranswidth}

   % The lead up to the transition
   \draw [#2]
         ([yshift=\vstart]last brick)
      -- ([yshift=\vstart, xshift=\thisleadwidth]last brick);

   % The transition itself
   \draw [#3]
         ([yshift=\vstart, xshift=\thisleadwidth]last brick)
      -- ([yshift=\vend, xshift=\transitionwidth]last brick)
      -- ++(#1*\wavewidth - \transitionwidth,0);

   \coordinate (last transition) at ([xshift=0.5*\transitionwidth]last brick);

   \advancebrick{#1}
}

% Generic curved bit-to-bit transition
%  #1: brick width
%  #2: Last brick style
%  #3: This brick style
%  #4: Last bit (offset from bottom)
%  #5: New bit (offset from bottom)
\newcommand{\brickcurvedbittobit}[5]{
   \pgfmathsetlengthmacro{\vstart}{(#4-0.5)*\waveheight}
   \pgfmathsetlengthmacro{\vend}{(#5-0.5)*\waveheight}

   % The curve itself
   \draw [#2]
         ([yshift=\vstart]last brick)
      .. controls ([yshift=\vstart]last brick)
              and ([yshift=\vend, xshift=0.2*\curvedtransitionwidth]last brick)
      .. ([yshift=\vend, xshift=\curvedtransitionwidth]last brick);

   % Start of the new bit
   \draw [#3]
         ([yshift=\vend, xshift=\curvedtransitionwidth]last brick)
      -- ++(#1*\wavewidth - \curvedtransitionwidth,0);

   \coordinate (last transition) at ([xshift=0.5*\curvedtransitionwidth]last brick);

   \advancebrick{#1}
}

% Generic soft transition from bit to bus
%  #1: brick width
%  #2: Bit style
%  #3: Bus style
%  #4: Bit (offset from bottom)
\newcommand{\bricksmoothbittobus}[4]{
   \pgfmathsetlengthmacro{\voffset}{(#4-0.5)*\waveheight}

   % Scale the transition depending on the magnitude of the level change
   \pgfmathsetlengthmacro{\thistranswidth}{(abs(#4-0.5)+0.5)*\transitionwidth}
   \pgfmathsetlengthmacro{\thisleadwidth}{\transitionwidth - \thistranswidth}

   % The lead up to the transition
   \draw [#2]
         ([yshift=\voffset]last brick)
      -- ([yshift=\voffset, xshift=\thisleadwidth]last brick);

   % Open-up the bus
   \draw [#3]
         ([xshift=#1*\wavewidth,    yshift= 0.5*\waveheight]last brick)
      -- ([xshift=\transitionwidth, yshift= 0.5*\waveheight]last brick)
      -- ([xshift=\thisleadwidth,   yshift=     \voffset]   last brick)
      -- ([xshift=\transitionwidth, yshift=-0.5*\waveheight]last brick)
      -- ([xshift=#1*\wavewidth,    yshift=-0.5*\waveheight]last brick)
      ;

   \coordinate (last transition) at ([xshift=0.5*\transitionwidth]last brick);

   \advancebrick{#1}
}

% Generic smooth transition from bus to bus
%  #1: brick width
%  #2: Bus style
%  #3: Bit style
\newcommand{\brickbustobus}[3]{

   % Close-down the old bus
   \draw [#2]
         ([yshift= 0.5*\waveheight]    last brick)
      -- ([xshift=0.5*\transitionwidth]last brick)
      -- ([yshift=-0.5*\waveheight]    last brick)
      ;

   % Open-up the new bus
   \draw [#3]
         ([xshift=#1*\wavewidth,    yshift= 0.5*\waveheight]last brick)
      -- ([xshift=\transitionwidth, yshift= 0.5*\waveheight]last brick)
      -- ([xshift=0.5*\transitionwidth]                     last brick)
      -- ([xshift=\transitionwidth, yshift=-0.5*\waveheight]last brick)
      -- ([xshift=#1*\wavewidth,    yshift=-0.5*\waveheight]last brick)
      ;

   \coordinate (last transition) at ([xshift=0.5*\transitionwidth]last brick);

   \advancebrick{#1}
}

% Generic smooth transition from bus to bit
%  #1: brick width
%  #2: Bus style
%  #3: Bit style
%  #4: Bit (offset from bottom)
\newcommand{\bricksmoothbustobit}[4]{
   \pgfmathsetlengthmacro{\voffset}{(#4-0.5)*\waveheight}

   % Scale the transition depending on the magnitude of the level change
   \pgfmathsetlengthmacro{\thistranswidth}{(abs(#4-0.5)+0.5)*\transitionwidth}
   \pgfmathsetlengthmacro{\thisleadwidth}{\transitionwidth - \thistranswidth}

   % Close-down the bus
   \draw [#2]
         ([                         yshift= 0.5*\waveheight]last brick)
      -- ([xshift=\thisleadwidth,   yshift= 0.5*\waveheight]last brick)
      -- ([xshift=\transitionwidth, yshift=     \voffset]   last brick)
      -- ([xshift=\thisleadwidth,   yshift=-0.5*\waveheight]last brick)
      -- ([                         yshift=-0.5*\waveheight]last brick)
      ;

   % The new bit
   \draw [#3]
         ([xshift=\transitionwidth, yshift=\voffset]last brick)
      -- ([xshift=#1*\wavewidth,    yshift=\voffset]last brick);

   \coordinate (last transition) at ([xshift=0.5*\transitionwidth]last brick);

   \advancebrick{#1}
}

% Generic curved transition from bus to bit
%  #1: brick width
%  #2: Bus style
%  #3: Bit style
%  #4: Bit (offset from bottom)
\newcommand{\brickcurvedbustobit}[4]{
   \pgfmathsetlengthmacro{\voffset}{(#4-0.5)*\waveheight}

   % Scale the transition depending on the magnitude of the level change
   \pgfmathsetlengthmacro{\thistranswidth}{(abs(#4-0.5)+0.5)*\transitionwidth}
   \pgfmathsetlengthmacro{\thisleadwidth}{\transitionwidth - \thistranswidth}

   % Close-down the bus
   \draw [#2]
         ([yshift= 0.5*\waveheight]last brick)
      .. controls ([                                   yshift= 0.5*\waveheight]last brick)
              and ([xshift=0.2*\curvedtransitionwidth, yshift=     \voffset]   last brick)
      .. ([xshift=\curvedtransitionwidth, yshift= \voffset]last brick)
      .. controls ([xshift=0.2*\curvedtransitionwidth, yshift=     \voffset]   last brick)
              and ([                                   yshift=-0.5*\waveheight]last brick)
      .. ([yshift=-0.5*\waveheight]last brick)
      ;

   % The new bit
   \draw [#3]
         ([xshift=\curvedtransitionwidth, yshift=\voffset]last brick)
      -- ([xshift=#1*\wavewidth,    yshift=\voffset]last brick);

   \coordinate (last transition) at ([xshift=0.5*\curvedtransitionwidth]last brick);

   \advancebrick{#1}
}

