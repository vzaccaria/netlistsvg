\makeatletter

% Inherits everything from rectangle except the behind background path.
%



\def\anchorsfromrectangle{%
  \inheritsavedanchors[from=rectangle]
  \inheritanchor[from=rectangle]{center}
  \inheritanchorborder[from=rectangle]
  \foreach \anchor in {north,north west,north east,center,west,east,mid, mid west,mid east,base,base west,base east,south,south west,south east}{ \inheritanchor[from=rectangle]{\anchor}}%
}

\def\anchorsfromtrapezium{%
  \inheritsavedanchors[from=trapezium]
  \inheritanchor[from=trapezium]{center}
  \inheritanchorborder[from=trapezium]
  \foreach \anchor in {bottom left corner, bottom right, top left corner, top right corner}
     { \inheritanchor[from=trapezium]{\anchor}}%
}

\def\absxfrom#1{\advance\pgf@x by #1}
\def\absyfrom#1{\advance\pgf@y by #1}
\def\relyfrom#1{
        \pgf@ya=#1\pgf@y%
        \advance\pgf@y by \pgf@ya%
}

\def\relxfrom#1{
        \pgf@xa=#1\pgf@x%
        \advance\pgf@x by \pgf@xa%
}

\pgfdeclareshape{alu}
{
  \anchorsfromtrapezium
    \inheritbackgroundpath[from=trapezium]
    \inheritbeforebackgroundpath[from=trapezium]
    \inheritbehindforegroundpath[from=trapezium]
    \inheritforegroundpath[from=trapezium]
    \inheritbeforeforegroundpath[from=trapezium]

    % parent saved macros cant be seen in the children when it is drawn
    % better create our own

    \savedanchor\myll{ \lowerleftpoint }
    \savedanchor\mylr{ \lowerrightpoint }
    \savedanchor\myur{ \upperrightpoint }

    \anchor{piar}{ \myll \relxfrom{-.6} }
    \anchor{pibr}{ \mylr \relxfrom{-.6}}
    \anchor{por} { \myur \pgf@x=0pt }
    \anchor{pia}{ \pgf@anchor@alu@piar\absyfrom{-5pt}}
    \anchor{pib}{ \pgf@anchor@alu@pibr\absyfrom{-5pt}}
    \anchor{po} { \pgf@anchor@alu@por \absyfrom{5pt}}

    \backgroundpath{%
        \def\vz@uc{\pgfpointlineattime{0.5}{\upperleftpoint}{\upperrightpoint}}%
        \def\vz@lc{\pgfpointlineattime{0.5}{\lowerleftpoint}{\lowerrightpoint}}%
        \def\vz@ic{\pgfpointlineattime{0.8}{\vz@uc}{\vz@lc}}%
        \def\vz@ac{\pgfpointlineattime{0.8}{\lowerleftpoint}{\vz@lc}}%
        \def\vz@bc{\pgfpointlineattime{0.8}{\lowerrightpoint}{\vz@lc}}%
        \color{black}%
        \pgfusepath{stroke} 

        % \pgfpathcircle{\pgf@anchor@alu@pia}{-2pt}
        % \pgfpathcircle{\pgf@anchor@alu@pib}{-2pt}
        % \pgfpathcircle{\pgf@anchor@alu@po}{2pt}
        % \pgfpathcircle{\vz@lc}{2pt}
        % \pgfpathcircle{\vz@ic}{2pt}
        % \pgfpathcircle{\vz@ac}{2pt}
        % \pgfpathcircle{\vz@bc}{2pt}

        \pgfpathmoveto{\lowerleftpoint}%
	\pgfpathlineto{\vz@ac}%
	\pgfpathlineto{\vz@ic}%
	\pgfpathlineto{\vz@bc}%
        \pgfpathlineto{\lowerrightpoint}%
        \pgfpathlineto{\lowerrightpoint}%
        \pgfpathlineto{\upperrightpoint}%
        \pgfpathlineto{\upperleftpoint}%
        \pgfpathclose
        
        \pgf@anchor@alu@center 
        %\pgftext[at={\pgfpoint{\pgf@x}{\pgf@y}}]{{\small ALU}}
                
        \connectanchors{\pgf@anchor@alu@piar}{\pgf@anchor@alu@pia}
        \connectanchors{\pgf@anchor@alu@pibr}{\pgf@anchor@alu@pib}
        \connectanchors{\pgf@anchor@alu@por}{\pgf@anchor@alu@po}
    }
}

\def\connectanchors#1#2{
     \pgfpathmoveto{#1}
     \pgfpathlineto{#2}%
     \color{black}%
     \pgfusepath{stroke}
}



\pgfdeclareshape{rfile}
{
  \anchorsfromrectangle
    \inheritbackgroundpath[from=rectangle]
    \inheritbeforebackgroundpath[from=rectangle]
    \inheritbehindforegroundpath[from=rectangle]
    \inheritforegroundpath[from=rectangle]
    \inheritbeforeforegroundpath[from=rectangle]

   \anchor{pir} { \southwest \pgf@y=0pt }
   \anchor{poar}{ \northeast\relyfrom{-.4}}
   \anchor{pobr}{ \pgf@anchor@rfile@poar\relyfrom{-2}}


   \anchor{pi}  { \pgf@anchor@rfile@pir\absxfrom{-5 pt}}
   \anchor{poa} { \pgf@anchor@rfile@poar\absxfrom{5 pt}}
   \anchor{pob} { \pgf@anchor@rfile@pobr\absxfrom{5 pt}}
   

   % Now do the background filling.
    \behindbackgroundpath{%
        \pgfextractx{\pgf@xa}{\southwest}%
        \pgfextracty{\pgf@ya}{\southwest}%
        \pgfextractx{\pgf@xb}{\northeast}%
        \pgfextracty{\pgf@yb}{\northeast}%

        \def\vz@nw{\pgfpoint{\pgf@xa}{\pgf@yb}}%
        \def\vz@se{\pgfpoint{\pgf@xb}{\pgf@ya}}%
        % inverting the following these does not work:
        %                                   ______|____________
        \def\vz@nc{\pgfpointlineattime{0.5}{\northeast}{\vz@nw}}%
        \def\vz@sc{\pgfpointlineattime{0.5}{\southwest}{\vz@se}}%

        % fill left part

        \pgfpathmoveto{\southwest}%
        \pgfpathlineto{\vz@nw}%
        \pgfpathlineto{\vz@nc}%
        \pgfpathlineto{\vz@sc}%
        \pgfpathclose
        \color{\pgf@left@color}%
        \pgfusepath{fill}%

        % fill right part
        \pgfpathmoveto{\northeast}%
        \pgfpathlineto{\vz@nc}%
        \pgfpathlineto{\vz@sc}%
        \pgfpathlineto{\vz@se}%
        \pgfpathclose
        \color{\pgf@right@color}%
        \pgfusepath{fill}%

        % draw from par to pa
        % dont use x, xa, y, yb .. as they are overwritten

        \ifpgf@rf@dontshowpins
                \connectanchors{\pgf@anchor@rfile@pi}{\pgf@anchor@rfile@pir}
                %dontdo anything
                \color{black}%
        \else
                \connectanchors{\pgf@anchor@rfile@poa}{\pgf@anchor@rfile@poar}
                \connectanchors{\pgf@anchor@rfile@pob}{\pgf@anchor@rfile@pobr}
                \connectanchors{\pgf@anchor@rfile@pi}{\pgf@anchor@rfile@pir}
        \fi
        \pgf@anchor@rfile@center
        \pgftext[at={\pgfpoint{\pgf@x}{\pgf@y}}]{{Reg}}
    }
}

\pgfdeclareshape{mem}
{
  \anchorsfromrectangle
    \inheritbackgroundpath[from=rectangle]
    \inheritbeforebackgroundpath[from=rectangle]
    \inheritbehindforegroundpath[from=rectangle]
    \inheritforegroundpath[from=rectangle]
    \inheritbeforeforegroundpath[from=rectangle]

   \anchor{pir} { \southwest \pgf@y=0pt }
   \anchor{por} { \northeast \pgf@y=0pt }

   \anchor{pi}  { \pgf@anchor@mem@pir\absxfrom{-5 pt}}
   \anchor{po}  { \pgf@anchor@mem@por\absxfrom{5 pt}}
   

   % Now do the background filling.
    \behindbackgroundpath{%
        \connectanchors{\pgf@anchor@mem@po}{\pgf@anchor@mem@por}
        \connectanchors{\pgf@anchor@mem@pi}{\pgf@anchor@mem@pir}
        \pgf@anchor@mem@center
        \pgftext[at={\pgfpoint{\pgf@x}{\pgf@y}}]{{\pgf@mem@label}}
    }
}

\newif\ifpgf@rf@dontshowpins
\def\pgf@right@color{gray!10}
\def\pgf@left@color{gray!30}
\def\pgf@mem@label{IM}

% Use these with PGF
\def\pgfsetleftcolor#1{\def\pgf@left@color{#1}}%
\def\pgfsetrightcolor#1{\def\pgf@right@color{#1}}%
\def\pgfsetrfdontshowpins{\pgf@rf@dontshowpinstrue}%
\def\pgfsetmemlabel#1{\def\pgf@mem@label{#1}}%

% Use these with TikZ
\tikzoption{mem label}{\pgfsetmemlabel{#1}}
\tikzoption{left color}{\pgfsetleftcolor{#1}}
\tikzoption{right color}{\pgfsetrightcolor{#1}}
\tikzoption{dont show pins}[]{\pgfsetrfdontshowpins}

\makeatother

\tikzset{ pipe/.pic = {
\node (-i) [shape=mem, draw, minimum width=7mm, minimum height=7mm] at (0,0) {};
\node (-d) [shape=rfile,draw,minimum width=7mm, minimum height=7mm, left color=gray!30, right color=white] at (1.5,0) {};
\node (-e) [alu, draw, minimum width=5mm, minimum height=5mm, rotate=-90] at (3,0) {};
\node (-m) [shape=mem, draw, minimum width=7mm, minimum height=7mm, inner sep=0, mem label=DM] at (4.5,0) {};
\node (-w) [shape=rfile,draw,minimum width=7mm, minimum height=7mm, left color=white, right color=gray!30, dont show pins] at (6,0) {};
\draw (-i.po) -- (-d.pi);
\draw (-d.poa) -- (-e.pia);
\draw (-d.pob) -- (-e.pib);
\draw (-e.po) -- (-m.pi);   
\draw (-m.po) -- (-w.pi);
\node (-fi) [rectangle, draw, minimum width=1mm, minimum height=8mm, text width=0cm, inner sep=0, fill=gray!20] at (0.75,0) {};
\node (-fd) [rectangle, draw, minimum width=1mm, minimum height=8mm, text width=0cm, inner sep=0, fill=gray!20] at (2.25,0) {};
\node (-fe) [rectangle, draw, minimum width=1mm, minimum height=8mm, text width=0cm, inner sep=0, fill=gray!20] at (3.75,0) {};
\node (-fm) [rectangle, draw, minimum width=1mm, minimum height=8mm, text width=0cm, inner sep=0, fill=gray!20] at (5.25,0) {};
}}

